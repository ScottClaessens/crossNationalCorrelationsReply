% Options for packages loaded elsewhere
\PassOptionsToPackage{unicode}{hyperref}
\PassOptionsToPackage{hyphens}{url}
%
\documentclass[
  man, donotrepeattitle,floatsintext]{apa6}
\usepackage{amsmath,amssymb}
\usepackage{iftex}
\ifPDFTeX
  \usepackage[T1]{fontenc}
  \usepackage[utf8]{inputenc}
  \usepackage{textcomp} % provide euro and other symbols
\else % if luatex or xetex
  \usepackage{unicode-math} % this also loads fontspec
  \defaultfontfeatures{Scale=MatchLowercase}
  \defaultfontfeatures[\rmfamily]{Ligatures=TeX,Scale=1}
\fi
\usepackage{lmodern}
\ifPDFTeX\else
  % xetex/luatex font selection
\fi
% Use upquote if available, for straight quotes in verbatim environments
\IfFileExists{upquote.sty}{\usepackage{upquote}}{}
\IfFileExists{microtype.sty}{% use microtype if available
  \usepackage[]{microtype}
  \UseMicrotypeSet[protrusion]{basicmath} % disable protrusion for tt fonts
}{}
\makeatletter
\@ifundefined{KOMAClassName}{% if non-KOMA class
  \IfFileExists{parskip.sty}{%
    \usepackage{parskip}
  }{% else
    \setlength{\parindent}{0pt}
    \setlength{\parskip}{6pt plus 2pt minus 1pt}}
}{% if KOMA class
  \KOMAoptions{parskip=half}}
\makeatother
\usepackage{xcolor}
\usepackage{graphicx}
\makeatletter
\def\maxwidth{\ifdim\Gin@nat@width>\linewidth\linewidth\else\Gin@nat@width\fi}
\def\maxheight{\ifdim\Gin@nat@height>\textheight\textheight\else\Gin@nat@height\fi}
\makeatother
% Scale images if necessary, so that they will not overflow the page
% margins by default, and it is still possible to overwrite the defaults
% using explicit options in \includegraphics[width, height, ...]{}
\setkeys{Gin}{width=\maxwidth,height=\maxheight,keepaspectratio}
% Set default figure placement to htbp
\makeatletter
\def\fps@figure{htbp}
\makeatother
\setlength{\emergencystretch}{3em} % prevent overfull lines
\providecommand{\tightlist}{%
  \setlength{\itemsep}{0pt}\setlength{\parskip}{0pt}}
\setcounter{secnumdepth}{-\maxdimen} % remove section numbering
% Make \paragraph and \subparagraph free-standing
\ifx\paragraph\undefined\else
  \let\oldparagraph\paragraph
  \renewcommand{\paragraph}[1]{\oldparagraph{#1}\mbox{}}
\fi
\ifx\subparagraph\undefined\else
  \let\oldsubparagraph\subparagraph
  \renewcommand{\subparagraph}[1]{\oldsubparagraph{#1}\mbox{}}
\fi
\newlength{\cslhangindent}
\setlength{\cslhangindent}{1.5em}
\newlength{\csllabelwidth}
\setlength{\csllabelwidth}{3em}
\newlength{\cslentryspacingunit} % times entry-spacing
\setlength{\cslentryspacingunit}{\parskip}
\newenvironment{CSLReferences}[2] % #1 hanging-ident, #2 entry spacing
 {% don't indent paragraphs
  \setlength{\parindent}{0pt}
  % turn on hanging indent if param 1 is 1
  \ifodd #1
  \let\oldpar\par
  \def\par{\hangindent=\cslhangindent\oldpar}
  \fi
  % set entry spacing
  \setlength{\parskip}{#2\cslentryspacingunit}
 }%
 {}
\usepackage{calc}
\newcommand{\CSLBlock}[1]{#1\hfill\break}
\newcommand{\CSLLeftMargin}[1]{\parbox[t]{\csllabelwidth}{#1}}
\newcommand{\CSLRightInline}[1]{\parbox[t]{\linewidth - \csllabelwidth}{#1}\break}
\newcommand{\CSLIndent}[1]{\hspace{\cslhangindent}#1}
\ifLuaTeX
\usepackage[bidi=basic]{babel}
\else
\usepackage[bidi=default]{babel}
\fi
\babelprovide[main,import]{english}
% get rid of language-specific shorthands (see #6817):
\let\LanguageShortHands\languageshorthands
\def\languageshorthands#1{}
% Manuscript styling
\usepackage{upgreek}
\captionsetup{font=singlespacing,justification=justified}

% Table formatting
\usepackage{longtable}
\usepackage{lscape}
% \usepackage[counterclockwise]{rotating}   % Landscape page setup for large tables
\usepackage{multirow}		% Table styling
\usepackage{tabularx}		% Control Column width
\usepackage[flushleft]{threeparttable}	% Allows for three part tables with a specified notes section
\usepackage{threeparttablex}            % Lets threeparttable work with longtable

% Create new environments so endfloat can handle them
% \newenvironment{ltable}
%   {\begin{landscape}\centering\begin{threeparttable}}
%   {\end{threeparttable}\end{landscape}}
\newenvironment{lltable}{\begin{landscape}\centering\begin{ThreePartTable}}{\end{ThreePartTable}\end{landscape}}

% Enables adjusting longtable caption width to table width
% Solution found at http://golatex.de/longtable-mit-caption-so-breit-wie-die-tabelle-t15767.html
\makeatletter
\newcommand\LastLTentrywidth{1em}
\newlength\longtablewidth
\setlength{\longtablewidth}{1in}
\newcommand{\getlongtablewidth}{\begingroup \ifcsname LT@\roman{LT@tables}\endcsname \global\longtablewidth=0pt \renewcommand{\LT@entry}[2]{\global\advance\longtablewidth by ##2\relax\gdef\LastLTentrywidth{##2}}\@nameuse{LT@\roman{LT@tables}} \fi \endgroup}

% \setlength{\parindent}{0.5in}
% \setlength{\parskip}{0pt plus 0pt minus 0pt}

% Overwrite redefinition of paragraph and subparagraph by the default LaTeX template
% See https://github.com/crsh/papaja/issues/292
\makeatletter
\renewcommand{\paragraph}{\@startsection{paragraph}{4}{\parindent}%
  {0\baselineskip \@plus 0.2ex \@minus 0.2ex}%
  {-1em}%
  {\normalfont\normalsize\bfseries\itshape\typesectitle}}

\renewcommand{\subparagraph}[1]{\@startsection{subparagraph}{5}{1em}%
  {0\baselineskip \@plus 0.2ex \@minus 0.2ex}%
  {-\z@\relax}%
  {\normalfont\normalsize\itshape\hspace{\parindent}{#1}\textit{\addperi}}{\relax}}
\makeatother

\makeatletter
\usepackage{etoolbox}
\patchcmd{\maketitle}
  {\section{\normalfont\normalsize\abstractname}}
  {\section*{\normalfont\normalsize\abstractname}}
  {}{\typeout{Failed to patch abstract.}}
\patchcmd{\maketitle}
  {\section{\protect\normalfont{\@title}}}
  {\section*{\protect\normalfont{\@title}}}
  {}{\typeout{Failed to patch title.}}
\makeatother

\usepackage{xpatch}
\makeatletter
\xapptocmd\appendix
  {\xapptocmd\section
    {\addcontentsline{toc}{section}{\appendixname\ifoneappendix\else~\theappendix\fi: #1}}
    {}{\InnerPatchFailed}%
  }
{}{\PatchFailed}
\makeatother
\usepackage{lineno}

\linenumbers
\usepackage{csquotes}
\raggedbottom
\usepackage{setspace}
\AtBeginEnvironment{tabular}{\singlespacing}
\AtBeginEnvironment{lltable}{\singlespacing}
\AtBeginEnvironment{tablenotes}{\doublespacing}
\captionsetup[table]{font={stretch=1.5}}
\captionsetup[figure]{font={stretch=1,small}}
\nolinenumbers
\ifLuaTeX
  \usepackage{selnolig}  % disable illegal ligatures
\fi
\IfFileExists{bookmark.sty}{\usepackage{bookmark}}{\usepackage{hyperref}}
\IfFileExists{xurl.sty}{\usepackage{xurl}}{} % add URL line breaks if available
\urlstyle{same}
\hypersetup{
  pdftitle={Reply to: Controlling for non-independence of nations should not be the default choice in cross-cultural research},
  pdfauthor={Scott Claessens1*, Thanos Kyritsis1, \& Quentin D. Atkinson1,2*},
  pdflang={en-EN},
  hidelinks,
  pdfcreator={LaTeX via pandoc}}

\title{Reply to: Controlling for non-independence of nations should not be the default choice in cross-cultural research}
\author{Scott Claessens\textsuperscript{1*}, Thanos Kyritsis\textsuperscript{1}, \& Quentin D. Atkinson\textsuperscript{1,2*}}
\date{}


\shorttitle{Non-independence reply}

\affiliation{\vspace{0.5cm}\textsuperscript{1} \footnotesize School of Psychology, University of Auckland, Auckland, New Zealand\\\textsuperscript{2} \footnotesize School of Anthropology and Museum Ethnography, University of Oxford, Oxford, United Kingdom \newline *email: \href{mailto:scott.claessens@gmail.com}{\nolinkurl{scott.claessens@gmail.com}}; \href{mailto:q.atkinson@auckland.ac.nz}{\nolinkurl{q.atkinson@auckland.ac.nz}}}

\begin{document}
\maketitle

\linenumbers

It is widely acknowledged that failing to account for non-independence in data
can bias statistical estimates and that national-level data often exhibit such
non-independence. In our recent paper\textsuperscript{1}, we show that, despite
this awareness, most analyses of cross-national variation in psychological
values or economic indicators do not make any attempt to control for
non-independence, and for those that do, most methods deployed do little to
reduce bias in simulated datasets with non-independence due to proximity or
shared cultural ancestry. We further show that reanalysing a small sample of
published datasets using the best performing methods from our simulations can
appreciably change parameter estimates.

In a commentary on our paper, Akaliyski\textsuperscript{2} highlights potential
issues with our study design, distinguishes between different sources of
confounding that can arise from non-independence, and proposes causal models
where controlling for non-independence might actually bias, rather than help,
inference. For these reasons, Akaliyski argues that controlling for
non-independence should not be the default analytic choice in cross-national
research.

We appreciate this commentary on our work. It is useful to discuss these
methodological issues in more detail to improve the rigour of cross-national
research. That said, we feel that the commentary misinterprets our
argument in places and overstates potential concerns.

It is worth clarifying from the outset that nowhere in the paper do we argue
that controlling for non-independence should be the default analytic choice
in cross-national analyses. Instead of proposing a one-size-fits-all
solution, we explicitly recommend that ``individual studies must outline
their own particular causal assumptions, which\ldots{} can then be used to design
tailored statistical estimation strategies''\textsuperscript{1} (p.~8). If controls
for non-independence are likely to bias inference, then they should not be
included. However, as we will see, we believe these scenarios to be much
rarer in the real world than Akaliyski claims.

The commentary raises concerns with our particular non-independence controls,
our simulations, and our reanalyses. First, while it is interesting that
Akaliyski\textsuperscript{3} finds a reduced effect of geographic and linguistic
distances in some regressions, our study and a large body of work\textsuperscript{4--11} shows that geography and language are important
explanatory factors for a range of cross-cultural variables. Second, in our
simulations we generated data with levels of spatial and cultural
non-independence that were comparable to the strength of non-independence
found in real-world datasets. We grant this does not prove that the controls
for non-independence will work in real-world scenarios, but such a proof
seems an impossibly high bar given that we can never know the true causal
model in the real world. Third, we acknowledge in the paper that our
reanalyses do not include all controls from the original studies, meaning
that ``we are unable to outright reject the claims from these studies''\textsuperscript{1} (p.~9). Our more modest goal was to show that adding
controls for non-independence in a stepwise fashion can have an appreciable
impact on reported relationships in real-world data.

Table 1 in the commentary distinguishes between two sources of
confounding that can arise from non-independence: confounding via diffusion
(the eponymous Galton's Problem) and confounding via third variables. This
is a useful contribution, as in hindsight we did not clearly delineate these
two sources of confounding. In the former scenario, we must control for
non-independence to account for local and historical diffusion of outcome
and predictor variables. In the latter scenario, we would like to control
for confounds such as climate, physical topography, cultural norms, and
institutions. If these variables are unobserved, then, to the extent that
they are autocorrelated in space or down cultural genealogies, we can
potentially use geographic and cultural phylogenetic distances to account
for their influence. Of course, if these third variables are observed, then
we should just include them directly: nowhere do we propose using controls
for non-independence as stand-ins for available direct measures. But if we
allow that unobserved confounds likely exist, then exploring the effect of
including controls for non-independence seems prudent, at the very least.

After defining different sources of confounding, Akaliyski proposes two
causal models where controlling for non-independence might actually bias
inference. Akaliyski's Models 4 and 5 illustrate scenarios where the predictor
variable (X) is non-independent, but the outcome variable (Y) is not. Under
such models, controlling for non-independence (Z) could either harm precision
or induce bias. Akaliyski argues that models like these may be more common
in reality than models where both variables are non-independent (Model 1)
because processes of diffusion are more likely to influence predictor
variables than outcome variables. If this is true, controls for
non-independence may harm inference more often than they will help.

However, compared to Model 1, Models 4 and 5 make much stronger and more
unrealistic assumptions about the causal relationships between
national-level variables. These models assume that spatial proximity and
cultural ancestry influence the outcome variable solely through the
predictor variable (Z → X → Y) and not through any other causal paths. This
seems to us an unlikely scenario. It is very likely that outcome variables
will also be affected by spatial proximity and cultural ancestry, either
directly or indirectly through their effects on unobserved variables.

We can illustrate this by expanding Akaliyski's example of collectivism
predicting mask wearing, using generative simulations to inform our intuitions
(Figure \ref{fig:plot}). If shared ancestry only influences mask
wearing through collectivism as a mediator, as Akaliyski claims, then
controlling for shared ancestry does slightly harm the precision of estimates
(Model 4a in Figure \ref{fig:plot}). But shared ancestry also likely
influences many other national-level variables that have been shown to be
related to COVID-19 preventative behaviours, including but not limited to
cultural tightness\textsuperscript{12}, long-term orientation\textsuperscript{13}, uncertainty
avoidance\textsuperscript{14}, religiosity\textsuperscript{15}, and trust in science
and government\textsuperscript{16}. If we have relevant data on all of these variables,
then we can control for them directly, but if any are unobserved then it is
necessary to control for shared ancestry to reduce bias (Model 4b in Figure
\ref{fig:plot}).



\begin{figure}
\centering
\includegraphics{manuscript_files/figure-latex/plot-1.pdf}
\caption{\label{fig:plot}Results of simulations testing the implications of controlling for non-independence under different generative causal models. In all simulations, the true causal effect of X on Y is 0. All other causal paths are set to 1. Z represents spatial or cultural phylogenetic non-independence. All other variables are standard normal variables. \emph{N} = 2000 in each simulated dataset. Densities show the posterior causal effect of X on Y from models fitted without controls for non-independence (red), with controls for non-independence (green), or with a control for non-independence included as an instrumental variable (blue). The IV control is only fitted to data generated from Model 5a, as Z only meets the criteria for an instrumental variable under this generative causal model.}
\end{figure}

Akaliyski then introduces an unobserved variable --- perceived pathogen threat
--- which influences both collectivism and mask wearing. Indeed, under this
generative model, controlling for shared ancestry results in bias
amplification\textsuperscript{17} (Model 5a in Figure \ref{fig:plot}).
But again, this causal model is unrealistic. If we allow that perceived pathogen
threat could be influenced by shared ancestry\textsuperscript{18} and that shared
ancestry could influence mask wearing indirectly through any of the unobserved
variables listed above, then it is necessary to control for shared ancestry to
reduce bias (though bias still remains due to the unblocked backdoor path
X ← U\textsubscript{1} → Y; Model 5b in Figure \ref{fig:plot}).

Even in situations where Models 4a or 5a do hold, we can still use controls for
non-independence to estimate unbiased causal effects. For example, in Model 5a,
Z fits the criteria for an instrumental variable: Z is independent of unmeasured
confounds and only influences Y through its effect on X\textsuperscript{19}. In this
situation, including Z as an instrumental variable adjusts for unmeasured
confounding, producing an unbiased estimate of the causal effect (Model 5a in
Figure \ref{fig:plot}).

In sum, while Akaliyski's commentary illustrates several theoretical cases
where controls for non-independence could hinder inferences, we contend that
such cases are likely to be rare in practice. This is because national-level
predictor and outcome variables (even contemporary variables) are likely to be
influenced by a host of factors that are spatially and culturally patterned.
Even in rare situations where such theoretical cases hold, controls for
non-independence can still be useful to include if implemented in a manner
consistent with the underlying causal assumptions (e.g., as instrumental
variables). Thus, we still think it critically important researchers consider
that cross-national analyses require additional controls to account for the
non-independence of nations.

\newpage
\nolinenumbers

\hypertarget{data-availability}{%
\section{Data Availability}\label{data-availability}}

All simulated data in this manuscript can be reproduced using the code on
GitHub: \url{https://github.com/ScottClaessens/crossNationalCorrelationsReply}

\hypertarget{code-availability}{%
\section{Code Availability}\label{code-availability}}

All code to reproduce the simulations in this manuscript can be found on
GitHub: \url{https://github.com/ScottClaessens/crossNationalCorrelationsReply}

\hypertarget{acknowledgements}{%
\section{Acknowledgements}\label{acknowledgements}}

This work was supported by a Royal Society of New Zealand Marsden grant
(20-UOA123) to QDA. We thank Erik Ringen for providing feedback on a previous
version of the manuscript and for his valuable input on instrumental variables.

\hypertarget{author-contributions-statement}{%
\section{Author Contributions Statement}\label{author-contributions-statement}}

SC and QDA wrote the original draft of the manuscript. SC ran the simulations
and created the figure. All authors reviewed and edited the final draft of the
manuscript.

\hypertarget{competing-interests-statement}{%
\section{Competing Interests Statement}\label{competing-interests-statement}}

The authors declare no competing interests.

\newpage

\hypertarget{references}{%
\section{References}\label{references}}

\begingroup

\hypertarget{refs}{}
\begin{CSLReferences}{0}{0}
\leavevmode\vadjust pre{\hypertarget{ref-Claessens2023}{}}%
\CSLLeftMargin{1. }%
\CSLRightInline{Claessens, S., Kyritsis, T. \& Atkinson, Q. D. \href{https://doi.org/10.1038/s41467-023-41486-1}{Cross-national analyses require additional controls to account for the non-independence of nations}. \emph{Nature Communications} \textbf{14}, 5776 (2023).}

\leavevmode\vadjust pre{\hypertarget{ref-Akaliyski2024}{}}%
\CSLLeftMargin{2. }%
\CSLRightInline{Akaliyski, P. Controlling for non-independence of nations should not be the default choice in crosscultural research: Response to {Claessens} et al. (2023). \emph{SSRN} (2024) doi:\href{https://doi.org/10.2139/ssrn.5014193}{10.2139/ssrn.5014193}.}

\leavevmode\vadjust pre{\hypertarget{ref-Akaliyski2017}{}}%
\CSLLeftMargin{3. }%
\CSLRightInline{Akaliyski, P. \href{https://doi.org/10.1163/15691330-12341432}{Sources of societal value similarities across {Europe}: Evidence from dyadic models}. \emph{Comparative Sociology} \textbf{16}, 447--470 (2017).}

\leavevmode\vadjust pre{\hypertarget{ref-Currie2021}{}}%
\CSLLeftMargin{4. }%
\CSLRightInline{Currie, T. E. \emph{et al.} \href{https://doi.org/10.1098/rstb.2020.0047}{The cultural evolution and ecology of institutions}. \emph{Philosophical Transactions of the Royal Society B: Biological Sciences} \textbf{376}, 20200047 (2021).}

\leavevmode\vadjust pre{\hypertarget{ref-Diamond1997}{}}%
\CSLLeftMargin{5. }%
\CSLRightInline{Diamond, J. M. \emph{Guns, germs, and steel: The fates of human societies}. (W. W. Norton \& Co., 1997).}

\leavevmode\vadjust pre{\hypertarget{ref-Dow2008}{}}%
\CSLLeftMargin{6. }%
\CSLRightInline{Dow, M. M. \& Eff, E. A. \href{https://doi.org/10.1177/1069397107311186}{Global, regional, and local network autocorrelation in the {Standard Cross-Cultural Sample}}. \emph{Cross-Cultural Research} \textbf{42}, 148--171 (2008).}

\leavevmode\vadjust pre{\hypertarget{ref-Gallup1999}{}}%
\CSLLeftMargin{7. }%
\CSLRightInline{Gallup, J. L., Sachs, J. D. \& Mellinger, A. D. \href{https://doi.org/10.1177/016001799761012334}{Geography and economic development}. \emph{International Regional Science Review} \textbf{22}, 179--232 (1999).}

\leavevmode\vadjust pre{\hypertarget{ref-Guglielmino1995}{}}%
\CSLLeftMargin{8. }%
\CSLRightInline{Guglielmino, C. R., Viganotti, C., Hewlett, B. \& Cavalli-Sforza, L. L. \href{https://doi.org/10.1073/pnas.92.16.7585}{Cultural variation in {Africa}: Role of mechanisms of transmission and adaptation}. \emph{Proceedings of the National Academy of Sciences} \textbf{92}, 7585--7589 (1995).}

\leavevmode\vadjust pre{\hypertarget{ref-Kelly2020}{}}%
\CSLLeftMargin{9. }%
\CSLRightInline{Kelly, M. \href{http://ssrn.com/abstract=3688200}{Understanding persistence}. \emph{CEPR Discussion Paper No. DP15246} (2020).}

\leavevmode\vadjust pre{\hypertarget{ref-Kyritsis2022}{}}%
\CSLLeftMargin{10. }%
\CSLRightInline{Kyritsis, T., Matthews, L. J., Welch, D. \& Atkinson, Q. D. \href{https://doi.org/10.1017/ehs.2022.40}{Shared cultural ancestry predicts the global diffusion of democracy}. \emph{Evolutionary Human Sciences} \textbf{4}, e42 (2022).}

\leavevmode\vadjust pre{\hypertarget{ref-Matthews2016}{}}%
\CSLLeftMargin{11. }%
\CSLRightInline{Matthews, L. J., Passmore, S., Richard, P. M., Gray, R. D. \& Atkinson, Q. D. \href{https://doi.org/10.1371/journal.pone.0152979}{Shared cultural history as a predictor of political and economic changes among nation states}. \emph{PLOS ONE} \textbf{11}, 1--18 (2016).}

\leavevmode\vadjust pre{\hypertarget{ref-Gilliam2022}{}}%
\CSLLeftMargin{12. }%
\CSLRightInline{Gilliam, A., Schwartz, D. B., Godoy, R., Boduroglu, A. \& Gutchess, A. \href{https://doi.org/10.1177/00220221221077710}{Does state tightness-looseness predict behavior and attitudes early in the {COVID-19} pandemic in the USA?} \emph{Journal of Cross-Cultural Psychology} \textbf{53}, 522--542 (2022).}

\leavevmode\vadjust pre{\hypertarget{ref-Ma2022}{}}%
\CSLLeftMargin{13. }%
\CSLRightInline{Ma, J.-T. \emph{et al.} \href{https://doi.org/10.1016/j.paid.2022.111589}{Long-term orientation and demographics predict the willingness to quarantine: A cross-national survey in the first round of {COVID-19} lockdown}. \emph{Personality and Individual Differences} \textbf{192}, 111589 (2022).}

\leavevmode\vadjust pre{\hypertarget{ref-Huang2023}{}}%
\CSLLeftMargin{14. }%
\CSLRightInline{Huang, X. \emph{et al.} \href{https://doi.org/10.1177/10693971221141478}{How national culture influences the speed of {COVID-19} spread: Three cross-cultural studies}. \emph{Cross-Cultural Research} \textbf{57}, 193--238 (2023).}

\leavevmode\vadjust pre{\hypertarget{ref-Trepanowski2022}{}}%
\CSLLeftMargin{15. }%
\CSLRightInline{Trepanowski, R. \& Drążkowski, D. \href{https://doi.org/10.1007/s10943-022-01569-7}{Cross-national comparison of religion as a predictor of {COVID-19} vaccination rates}. \emph{Journal of Religion and Health} \textbf{61}, 2198--2211 (2022).}

\leavevmode\vadjust pre{\hypertarget{ref-Chen2022}{}}%
\CSLLeftMargin{16. }%
\CSLRightInline{Chen, R., Fwu, B.-J., Yang, T.-R., Chen, Y.-K. \& Tran, Q.-A. N. \href{https://doi.org/10.1371/journal.pone.0270160}{To mask or not to mask: Debunking the myths of mask-wearing during {COVID-19} across cultures}. \emph{PLOS ONE} \textbf{17}, 1--17 (2022).}

\leavevmode\vadjust pre{\hypertarget{ref-Cinelli2022}{}}%
\CSLLeftMargin{17. }%
\CSLRightInline{Cinelli, C., Forney, A. \& Pearl, J. A crash course in good and bad controls. \emph{Sociological Methods \& Research} 00491241221099552 (2022) doi:\href{https://doi.org/10.1177/00491241221099552}{10.1177/00491241221099552}.}

\leavevmode\vadjust pre{\hypertarget{ref-Bromham2018}{}}%
\CSLLeftMargin{18. }%
\CSLRightInline{Bromham, L., Hua, X., Cardillo, M., Schneemann, H. \& Greenhill, S. J. \href{https://doi.org/10.1098/rsos.181100}{Parasites and politics: Why cross-cultural studies must control for relatedness, proximity and covariation}. \emph{Royal Society Open Science} \textbf{5}, 181100 (2018).}

\leavevmode\vadjust pre{\hypertarget{ref-McElreath2020}{}}%
\CSLLeftMargin{19. }%
\CSLRightInline{McElreath, R. \emph{Statistical rethinking: A {B}ayesian course with examples in {R} and {Stan}}. (CRC Press, 2020).}

\end{CSLReferences}

\endgroup


\end{document}
